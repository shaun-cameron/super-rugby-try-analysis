% Options for packages loaded elsewhere
\PassOptionsToPackage{unicode}{hyperref}
\PassOptionsToPackage{hyphens}{url}
%
\documentclass[
]{article}
\usepackage{lmodern}
\usepackage{amssymb,amsmath}
\usepackage{ifxetex,ifluatex}
\ifnum 0\ifxetex 1\fi\ifluatex 1\fi=0 % if pdftex
  \usepackage[T1]{fontenc}
  \usepackage[utf8]{inputenc}
  \usepackage{textcomp} % provide euro and other symbols
\else % if luatex or xetex
  \usepackage{unicode-math}
  \defaultfontfeatures{Scale=MatchLowercase}
  \defaultfontfeatures[\rmfamily]{Ligatures=TeX,Scale=1}
\fi
% Use upquote if available, for straight quotes in verbatim environments
\IfFileExists{upquote.sty}{\usepackage{upquote}}{}
\IfFileExists{microtype.sty}{% use microtype if available
  \usepackage[]{microtype}
  \UseMicrotypeSet[protrusion]{basicmath} % disable protrusion for tt fonts
}{}
\makeatletter
\@ifundefined{KOMAClassName}{% if non-KOMA class
  \IfFileExists{parskip.sty}{%
    \usepackage{parskip}
  }{% else
    \setlength{\parindent}{0pt}
    \setlength{\parskip}{6pt plus 2pt minus 1pt}}
}{% if KOMA class
  \KOMAoptions{parskip=half}}
\makeatother
\usepackage{xcolor}
\IfFileExists{xurl.sty}{\usepackage{xurl}}{} % add URL line breaks if available
\IfFileExists{bookmark.sty}{\usepackage{bookmark}}{\usepackage{hyperref}}
\hypersetup{
  pdftitle={Super Rugby Try Analysis},
  pdfauthor={Shaun Cameron},
  hidelinks,
  pdfcreator={LaTeX via pandoc}}
\urlstyle{same} % disable monospaced font for URLs
\usepackage[margin=1in]{geometry}
\usepackage{color}
\usepackage{fancyvrb}
\newcommand{\VerbBar}{|}
\newcommand{\VERB}{\Verb[commandchars=\\\{\}]}
\DefineVerbatimEnvironment{Highlighting}{Verbatim}{commandchars=\\\{\}}
% Add ',fontsize=\small' for more characters per line
\usepackage{framed}
\definecolor{shadecolor}{RGB}{248,248,248}
\newenvironment{Shaded}{\begin{snugshade}}{\end{snugshade}}
\newcommand{\AlertTok}[1]{\textcolor[rgb]{0.94,0.16,0.16}{#1}}
\newcommand{\AnnotationTok}[1]{\textcolor[rgb]{0.56,0.35,0.01}{\textbf{\textit{#1}}}}
\newcommand{\AttributeTok}[1]{\textcolor[rgb]{0.77,0.63,0.00}{#1}}
\newcommand{\BaseNTok}[1]{\textcolor[rgb]{0.00,0.00,0.81}{#1}}
\newcommand{\BuiltInTok}[1]{#1}
\newcommand{\CharTok}[1]{\textcolor[rgb]{0.31,0.60,0.02}{#1}}
\newcommand{\CommentTok}[1]{\textcolor[rgb]{0.56,0.35,0.01}{\textit{#1}}}
\newcommand{\CommentVarTok}[1]{\textcolor[rgb]{0.56,0.35,0.01}{\textbf{\textit{#1}}}}
\newcommand{\ConstantTok}[1]{\textcolor[rgb]{0.00,0.00,0.00}{#1}}
\newcommand{\ControlFlowTok}[1]{\textcolor[rgb]{0.13,0.29,0.53}{\textbf{#1}}}
\newcommand{\DataTypeTok}[1]{\textcolor[rgb]{0.13,0.29,0.53}{#1}}
\newcommand{\DecValTok}[1]{\textcolor[rgb]{0.00,0.00,0.81}{#1}}
\newcommand{\DocumentationTok}[1]{\textcolor[rgb]{0.56,0.35,0.01}{\textbf{\textit{#1}}}}
\newcommand{\ErrorTok}[1]{\textcolor[rgb]{0.64,0.00,0.00}{\textbf{#1}}}
\newcommand{\ExtensionTok}[1]{#1}
\newcommand{\FloatTok}[1]{\textcolor[rgb]{0.00,0.00,0.81}{#1}}
\newcommand{\FunctionTok}[1]{\textcolor[rgb]{0.00,0.00,0.00}{#1}}
\newcommand{\ImportTok}[1]{#1}
\newcommand{\InformationTok}[1]{\textcolor[rgb]{0.56,0.35,0.01}{\textbf{\textit{#1}}}}
\newcommand{\KeywordTok}[1]{\textcolor[rgb]{0.13,0.29,0.53}{\textbf{#1}}}
\newcommand{\NormalTok}[1]{#1}
\newcommand{\OperatorTok}[1]{\textcolor[rgb]{0.81,0.36,0.00}{\textbf{#1}}}
\newcommand{\OtherTok}[1]{\textcolor[rgb]{0.56,0.35,0.01}{#1}}
\newcommand{\PreprocessorTok}[1]{\textcolor[rgb]{0.56,0.35,0.01}{\textit{#1}}}
\newcommand{\RegionMarkerTok}[1]{#1}
\newcommand{\SpecialCharTok}[1]{\textcolor[rgb]{0.00,0.00,0.00}{#1}}
\newcommand{\SpecialStringTok}[1]{\textcolor[rgb]{0.31,0.60,0.02}{#1}}
\newcommand{\StringTok}[1]{\textcolor[rgb]{0.31,0.60,0.02}{#1}}
\newcommand{\VariableTok}[1]{\textcolor[rgb]{0.00,0.00,0.00}{#1}}
\newcommand{\VerbatimStringTok}[1]{\textcolor[rgb]{0.31,0.60,0.02}{#1}}
\newcommand{\WarningTok}[1]{\textcolor[rgb]{0.56,0.35,0.01}{\textbf{\textit{#1}}}}
\usepackage{graphicx,grffile}
\makeatletter
\def\maxwidth{\ifdim\Gin@nat@width>\linewidth\linewidth\else\Gin@nat@width\fi}
\def\maxheight{\ifdim\Gin@nat@height>\textheight\textheight\else\Gin@nat@height\fi}
\makeatother
% Scale images if necessary, so that they will not overflow the page
% margins by default, and it is still possible to overwrite the defaults
% using explicit options in \includegraphics[width, height, ...]{}
\setkeys{Gin}{width=\maxwidth,height=\maxheight,keepaspectratio}
% Set default figure placement to htbp
\makeatletter
\def\fps@figure{htbp}
\makeatother
\setlength{\emergencystretch}{3em} % prevent overfull lines
\providecommand{\tightlist}{%
  \setlength{\itemsep}{0pt}\setlength{\parskip}{0pt}}
\setcounter{secnumdepth}{-\maxdimen} % remove section numbering

\title{Super Rugby Try Analysis}
\author{Shaun Cameron}
\date{12/04/2020}

\begin{document}
\maketitle

\begin{Shaded}
\begin{Highlighting}[]
\KeywordTok{library}\NormalTok{(tinytex)}
\end{Highlighting}
\end{Shaded}

\hypertarget{the-data}{%
\section{\texorpdfstring{\textbf{The Data}}{The Data}}\label{the-data}}

The data used for this activity is stored in the
2017\_super-rugby\_try-source-data.csv file. This data consists of tries
that were scored during the 2017 Super Rugby competition
(observations/rows). Here is a description of the variables:

\textbf{try\_no:} a unique identification number given to each try

\textbf{round\_no:} an identification number to distinguish the round
number the try was scored in

\textbf{attacking\_team:} the try-scoring team

\textbf{defending\_team:} the opposition team who conceded the try

\textbf{attacking\_rank:} the final league ranking at the end of the
season of the try-scoring team

\textbf{defending\_rank:} the final league ranking at the end of the
season of the opposition team

\textbf{attacking\_conference:} the conference group of the try-scoring
team

\textbf{defending\_conference:} the conference group of the opposition
team

\textbf{game\_time:} the game time in minutes when the try was scored

\textbf{try\_source:} the initial source of possession for the attacking
team preceding the try

\textbf{final\_source:} the event that directly preceded the try and
resulted in the try being scored

\textbf{phases:} the total number of phases between gaining possession,
and the try being scored (a phase is from one ruck to the next ruck)

\textbf{time\_from\_source:} the time taken from gaining possession to
scoring the try, in seconds

\textbf{possession\_zone:} the zone on the field the attacking team
gained possession of the ball before scoring the try (A = attacking 22m
line to try-line, B = halfway to attacking 22m line, C = defensive 22m
line to halfway, D = ) \textbf{offloads:} the number of offloads from
gaining possession to the try being scored

\textbf{passes:} the number of passes from gaining possession to the try
being scored

\textbf{total\_passes:} the number of offloads plus passes

This data was collected by a former UC student, Molly Coughlan, as part
of a project that identified playing patterns that led to tries in super
rugby.

\hypertarget{r-setup}{%
\section{\texorpdfstring{\textbf{R Setup}}{R Setup}}\label{r-setup}}

\begin{Shaded}
\begin{Highlighting}[]
\KeywordTok{library}\NormalTok{(tidyverse) }\CommentTok{#load relevant packages}

\NormalTok{df <-}\StringTok{ }\KeywordTok{read_csv}\NormalTok{(}\StringTok{"data/2017_super-rugby_try-source-data.csv"}\NormalTok{) }\CommentTok{# load data source}
\end{Highlighting}
\end{Shaded}

\hypertarget{checking-data}{%
\section{\texorpdfstring{\textbf{Checking
Data}}{Checking Data}}\label{checking-data}}

\begin{Shaded}
\begin{Highlighting}[]
\KeywordTok{str}\NormalTok{(df) }\CommentTok{# provides structure of df}


\KeywordTok{head}\NormalTok{(df) }\CommentTok{# shows first 6 rows of df}


\KeywordTok{tail}\NormalTok{(df) }\CommentTok{# shows last 6 rows of df}
\end{Highlighting}
\end{Shaded}

\hypertarget{data-visualisation}{%
\section{\texorpdfstring{\textbf{Data
Visualisation}}{Data Visualisation}}\label{data-visualisation}}

\hypertarget{attacking-prowess}{%
\subsection{\texorpdfstring{\textbf{Attacking
Prowess}}{Attacking Prowess}}\label{attacking-prowess}}

This graph shows how many tries each team has scored.

\begin{Shaded}
\begin{Highlighting}[]
\KeywordTok{ggplot}\NormalTok{(}\DataTypeTok{data =}\NormalTok{ df) }\OperatorTok{+}
\StringTok{  }\KeywordTok{geom_bar}\NormalTok{(}\DataTypeTok{mapping =} \KeywordTok{aes}\NormalTok{(}\DataTypeTok{x =}\NormalTok{ attacking_team), }\DataTypeTok{stat =} \StringTok{"count"}\NormalTok{, }\DataTypeTok{fill =} \StringTok{"red"}\NormalTok{, }\DataTypeTok{colour =} \StringTok{"black"}\NormalTok{) }\OperatorTok{+}\StringTok{ }\KeywordTok{theme}\NormalTok{(}\DataTypeTok{axis.text.x =} \KeywordTok{element_text}\NormalTok{(}\DataTypeTok{angle =} \DecValTok{60}\NormalTok{, }\DataTypeTok{hjust =} \DecValTok{1}\NormalTok{))}
\end{Highlighting}
\end{Shaded}

\includegraphics{super-rugby-try-analysis_files/figure-latex/attacking prowess-1.pdf}

\hypertarget{when-were-the-tries-scored}{%
\subsection{\texorpdfstring{\textbf{When were the Tries
Scored?}}{When were the Tries Scored?}}\label{when-were-the-tries-scored}}

This graph shows what time the tries were scored. It seems as though
there is no real pattern as to when the tries are scored, although it is
interesting to note that the end of the first half (40) ranks among the
highest frequency. Perhaps this is due to the fact that play only stops
when the ball goes out or points are scored, and crossing the line is
the most beneficial play to make.

\begin{Shaded}
\begin{Highlighting}[]
\KeywordTok{ggplot}\NormalTok{(}\DataTypeTok{data =}\NormalTok{ df, }\KeywordTok{aes}\NormalTok{(}\DataTypeTok{x =}\NormalTok{ game_time)) }\OperatorTok{+}
\StringTok{  }\KeywordTok{geom_histogram}\NormalTok{(}\DataTypeTok{mapping =} \KeywordTok{aes}\NormalTok{(}\DataTypeTok{y =}\NormalTok{ ..density..), }\DataTypeTok{colour =} \StringTok{"dodgerblue"}\NormalTok{, }\DataTypeTok{fill =} \StringTok{"gold"}\NormalTok{) }\OperatorTok{+}
\StringTok{  }\KeywordTok{geom_density}\NormalTok{(}\DataTypeTok{alpha =} \FloatTok{0.3}\NormalTok{, }\DataTypeTok{fill =} \StringTok{"white"}\NormalTok{)}
\end{Highlighting}
\end{Shaded}

\begin{verbatim}
## `stat_bin()` using `bins = 30`. Pick better value with `binwidth`.
\end{verbatim}

\includegraphics{super-rugby-try-analysis_files/figure-latex/try times-1.pdf}

\hypertarget{sourcing-the-tries}{%
\subsection{\texorpdfstring{\textbf{Sourcing the
Tries}}{Sourcing the Tries}}\label{sourcing-the-tries}}

These graphs show the initial event of attacking possession leading to
the try and the final play before the try was scored respectively. It
makes sense that the lineout was the most common phase of play starting
the possession, given that most of the time in rugby the ball is kicked
out for positional advantage, and the lineout is usually the best option
for restarting play. In the final phase before the try is scored
however, the trends are a little bit less discernable. Multiphase plays
lead to the most tries, but they obviously come from a range of
different plays (hence the name).

\begin{Shaded}
\begin{Highlighting}[]
\KeywordTok{ggplot}\NormalTok{(df, }\KeywordTok{aes}\NormalTok{(}\DataTypeTok{x =}\NormalTok{ try_source), }\DataTypeTok{stat =} \StringTok{"count"}\NormalTok{) }\OperatorTok{+}
\StringTok{  }\KeywordTok{geom_bar}\NormalTok{(}\DataTypeTok{fill =} \StringTok{"red"}\NormalTok{, }\DataTypeTok{colour =} \StringTok{"green"}\NormalTok{) }\OperatorTok{+}\StringTok{ }\KeywordTok{theme}\NormalTok{(}\DataTypeTok{axis.text.x =} \KeywordTok{element_text}\NormalTok{(}\DataTypeTok{angle =} \DecValTok{60}\NormalTok{, }\DataTypeTok{hjust =} \DecValTok{1}\NormalTok{))}
\end{Highlighting}
\end{Shaded}

\includegraphics{super-rugby-try-analysis_files/figure-latex/try sources-1.pdf}

\begin{Shaded}
\begin{Highlighting}[]
\KeywordTok{ggplot}\NormalTok{(df, }\KeywordTok{aes}\NormalTok{(}\DataTypeTok{x =}\NormalTok{ final_source), }\DataTypeTok{stat =} \StringTok{"count"}\NormalTok{) }\OperatorTok{+}
\StringTok{  }\KeywordTok{geom_bar}\NormalTok{(}\DataTypeTok{fill =} \StringTok{"steelblue"}\NormalTok{, }\DataTypeTok{colour =} \StringTok{"slategrey"}\NormalTok{) }\OperatorTok{+}\StringTok{ }
\StringTok{  }\KeywordTok{theme}\NormalTok{(}\DataTypeTok{axis.text.x =} \KeywordTok{element_text}\NormalTok{(}\DataTypeTok{angle =} \DecValTok{60}\NormalTok{, }\DataTypeTok{hjust =} \DecValTok{1}\NormalTok{))}
\end{Highlighting}
\end{Shaded}

\includegraphics{super-rugby-try-analysis_files/figure-latex/try sources-2.pdf}

\hypertarget{where-did-the-play-begin}{%
\subsection{\texorpdfstring{\textbf{Where Did the Play
Begin?}}{Where Did the Play Begin?}}\label{where-did-the-play-begin}}

This graph looks at how far out the attacking team first gained
possession that resulted in a try. This graph makes perfect sense as the
highest proportion of tries occur when possession is first received with
22 metres of the tryline (A). The downward trend follows with B (22-50
metres out), C (halfway to defensive 22) and D (22 to defensive
tryline).

\begin{Shaded}
\begin{Highlighting}[]
\NormalTok{df }\OperatorTok
\StringTok{    }\KeywordTok{count}\NormalTok{(possession_zone) }\OperatorTok
\StringTok{    }\KeywordTok{mutate}\NormalTok{(}\DataTypeTok{prop =} \KeywordTok{round}\NormalTok{(}\KeywordTok{prop.table}\NormalTok{(n), }\DecValTok{3}\NormalTok{)) }\OperatorTok
\StringTok{    }\KeywordTok{ggplot}\NormalTok{() }\OperatorTok{+}\StringTok{ }
\StringTok{    }\KeywordTok{geom_bar}\NormalTok{(}\DataTypeTok{mapping =} \KeywordTok{aes}\NormalTok{(}\DataTypeTok{x =}\NormalTok{ possession_zone, }\DataTypeTok{y =}\NormalTok{ prop), }\DataTypeTok{colour =} \StringTok{"royalblue"}\NormalTok{, }\DataTypeTok{fill =} \StringTok{"salmon"}\NormalTok{, }\DataTypeTok{stat =} \StringTok{"identity"}\NormalTok{)}
\end{Highlighting}
\end{Shaded}

\includegraphics{super-rugby-try-analysis_files/figure-latex/possession zone-1.pdf}

\hypertarget{time-in-possession}{%
\subsection{\texorpdfstring{\textbf{Time in
Possession}}{Time in Possession}}\label{time-in-possession}}

The following graph looks at how long a team held possession before
scoring. It looks like having the ball for less than a minute is much
more ideal than holding it for longer, which is interesting. I suppose
looking to get the ball up the field as quickly as possible is much more
favourable than wearing the opposition down with multiple defensive
efforts.

\begin{Shaded}
\begin{Highlighting}[]
\KeywordTok{ggplot}\NormalTok{(df, }\KeywordTok{aes}\NormalTok{(}\DataTypeTok{x =}\NormalTok{ time_from_source)) }\OperatorTok{+}\StringTok{ }
\StringTok{  }\KeywordTok{geom_histogram}\NormalTok{(}\DataTypeTok{mapping =} \KeywordTok{aes}\NormalTok{(}\DataTypeTok{y =}\NormalTok{ ..density..), }\DataTypeTok{colour =} \StringTok{"tan1"}\NormalTok{, }\DataTypeTok{fill =} \StringTok{"brown"}\NormalTok{, }\DataTypeTok{binwidth =} \DecValTok{10}\NormalTok{) }\OperatorTok{+}
\StringTok{  }\KeywordTok{geom_density}\NormalTok{(}\DataTypeTok{alpha =} \FloatTok{0.3}\NormalTok{, }\DataTypeTok{fill =} \StringTok{"grey"}\NormalTok{)}
\end{Highlighting}
\end{Shaded}

\includegraphics{super-rugby-try-analysis_files/figure-latex/possession time-1.pdf}

\hypertarget{relationship-between-passing-phase-length-in-tryscoring}{%
\subsection{\texorpdfstring{\textbf{Relationship Between Passing \&
Phase Length in
Tryscoring}}{Relationship Between Passing \& Phase Length in Tryscoring}}\label{relationship-between-passing-phase-length-in-tryscoring}}

This graph highlights a strong relationship between the fact that the
more phases an attacking team has the more passes they are likely to
complete.

\begin{Shaded}
\begin{Highlighting}[]
\NormalTok{df }\OperatorTok
\StringTok{   }\KeywordTok{ggplot}\NormalTok{(}\DataTypeTok{mapping =} \KeywordTok{aes}\NormalTok{(}\DataTypeTok{x =}\NormalTok{ total_passes, }\DataTypeTok{y =}\NormalTok{ phases)) }\OperatorTok{+}
\StringTok{     }\KeywordTok{geom_point}\NormalTok{(}\DataTypeTok{alpha =} \FloatTok{0.3}\NormalTok{, }\DataTypeTok{size =} \DecValTok{1}\NormalTok{, }\DataTypeTok{colour =} \StringTok{"red"}\NormalTok{) }\OperatorTok{+}
\StringTok{     }\KeywordTok{theme}\NormalTok{(}\DataTypeTok{plot.background =} \KeywordTok{element_rect}\NormalTok{(}\DataTypeTok{fill =} \StringTok{"white"}\NormalTok{),}
           \DataTypeTok{panel.background =} \KeywordTok{element_rect}\NormalTok{(}\DataTypeTok{fill =} \StringTok{"lightcyan"}\NormalTok{),}
           \DataTypeTok{axis.line =} \KeywordTok{element_line}\NormalTok{(}\DataTypeTok{colour =} \StringTok{"grey"}\NormalTok{)) }\OperatorTok{+}\StringTok{ }
\StringTok{  }\KeywordTok{geom_smooth}\NormalTok{()}
\end{Highlighting}
\end{Shaded}

\begin{verbatim}
## `geom_smooth()` using method = 'loess' and formula 'y ~ x'
\end{verbatim}

\includegraphics{super-rugby-try-analysis_files/figure-latex/time pass data-1.pdf}

\hypertarget{team-try-totals}{%
\subsection{\texorpdfstring{\textbf{Team Try
Totals}}{Team Try Totals}}\label{team-try-totals}}

Graph that simply outlines the number of tries each team scored, colour
coded based on the colours of their playing strip.

\begin{Shaded}
\begin{Highlighting}[]
\KeywordTok{ggplot}\NormalTok{(df, }\KeywordTok{aes}\NormalTok{(}\DataTypeTok{x =}\NormalTok{ attacking_team, }\DataTypeTok{fill =}\NormalTok{ attacking_team, }\DataTypeTok{colour =}\NormalTok{ attacking_team)) }\OperatorTok{+}
\StringTok{    }\KeywordTok{geom_bar}\NormalTok{(}\DataTypeTok{stat =} \StringTok{"count"}\NormalTok{, }\DataTypeTok{position =} \StringTok{"dodge"}\NormalTok{) }\OperatorTok{+}\StringTok{ }
\StringTok{    }\KeywordTok{theme}\NormalTok{(}\DataTypeTok{axis.text.x =} \KeywordTok{element_text}\NormalTok{(}\DataTypeTok{angle =} \DecValTok{60}\NormalTok{)) }\OperatorTok{+}
\StringTok{    }\KeywordTok{scale_fill_manual}\NormalTok{(}\DataTypeTok{values =} \KeywordTok{c}\NormalTok{(}\StringTok{"dodgerblue4"}\NormalTok{, }\StringTok{"midnightblue"}\NormalTok{, }\StringTok{"steelblue3"}\NormalTok{, }\StringTok{"tan1"}\NormalTok{, }\StringTok{"black"}\NormalTok{, }\StringTok{"firebrick1"}\NormalTok{, }\StringTok{"mediumblue"}\NormalTok{, }\StringTok{"royalblue4"}\NormalTok{, }\StringTok{"black"}\NormalTok{, }\StringTok{"black"}\NormalTok{, }\StringTok{"black"}\NormalTok{, }\StringTok{"red1"}\NormalTok{, }\StringTok{"navyblue"}\NormalTok{, }\StringTok{"maroon"}\NormalTok{, }\StringTok{"black"}\NormalTok{, }\StringTok{"royalblue2"}\NormalTok{, }\StringTok{"red"}\NormalTok{, }\StringTok{"lightskyblue"}\NormalTok{)) }\OperatorTok{+}
\StringTok{    }\KeywordTok{scale_colour_manual}\NormalTok{(}\DataTypeTok{values =} \KeywordTok{c}\NormalTok{(}\StringTok{"steelblue1"}\NormalTok{, }\StringTok{"gold"}\NormalTok{, }\StringTok{"steelblue4"}\NormalTok{, }\StringTok{"white"}\NormalTok{, }\StringTok{"gold"}\NormalTok{, }\StringTok{"black"}\NormalTok{, }\StringTok{"black"}\NormalTok{, }\StringTok{"gold"}\NormalTok{, }\StringTok{"yellow"}\NormalTok{, }\StringTok{"orange"}\NormalTok{, }\StringTok{"firebrick1"}\NormalTok{, }\StringTok{"white"}\NormalTok{, }\StringTok{"red"}\NormalTok{, }\StringTok{"midnightblue"}\NormalTok{, }\StringTok{"white"}\NormalTok{, }\StringTok{"white"}\NormalTok{, }\StringTok{"white"}\NormalTok{, }\StringTok{"midnightblue"}\NormalTok{))}
\end{Highlighting}
\end{Shaded}

\includegraphics{super-rugby-try-analysis_files/figure-latex/team tries-1.pdf}

\end{document}
